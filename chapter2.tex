\setcounter{section}{1}
\section{Einleitung}

\begin{karte}{Metrik}
    Sei \(X\) eine Menge. Eine Funktion \(\abb{f}{X\times X}{\R_{\geq 0}}\) ist eine \textit{Metrik} 
    (oder Abstandsfunktion), falls für alle \(x,y,z\in X\) gilt: 
    \begin{description}
        \item[positiv definit] \(d(x,y) = 0 \Leftrightarrow x = y\),
        \item[symmetrisch] \(d(x,y) = d(y,x)\),
        \item[Dreiecksungleichung] \(d(x,z) \leq d(x,y) + d(y,z)\).   
    \end{description}
    Ein Paar \((X,d)\) bestehend aus einer Menge \(X\) versehen mit einer Metrik \(d\) heißt 
    \textit{metrischer Raum}.\\
    Die Teilmenge \(B_r(x) := \set{y\in X \;|\; d(x,y) \leq r}\) heißt abgeschlossener Ball in \(X\) 
    mit Zentrum \(x\) und Radius \(r\).\\
    Seien \((X,d_X)\) und \((Y,d_Y)\) metrische Räume. Eine Abbildung \(\abb{f}{X}{Y}\) heißt 
    \textit{abstandserhaltend}, falls für alle \(x,y\in X\) gilt: \(d_Y(f(x), f(y)) = d_X(x,y)\).\\
    Eine \textit{Isometrie} ist eine bijektive und abstandserhaltende Abbildung. Zwei metrische 
    Räume \(X\) und \(Y\) heißen isometrisch, falls es eine Isometrie \(\abb{f}{X}{Y}\) gibt.
\end{karte}

\begin{karte}{Euklidische Länge}
    Sei \(\abb{c}{[a,b]}{\R^2}, t\mapsto c(t) = (x(t),y(t))\) eine stückweise differenzierbare Kurve in 
    der Ebene. Die \textit{euklidische Länge} von \(c\) ist definiert als 
    \[ L_{euk}(c) := \int_a^b ||\frac{dc(t)}{dt}||dt = \int_a^b ||c'(t)||dt = \int_a^b \sqrt{x'(t)^2 + y'(t)^2} dt. \]

    \begin{enumerate}
        \item Die euklidische Länge \(L_{euk}(c)\) einer Kurve \(c\) ist unabhängig von der Parametrisierung.
        \item \(L_{euk}(c)\) ist invariant unter Translationen, Drehungen und Spiegelungen von \(\R^2\) 
        (also invariant unter den euklidischen Isometrien der Ebene).
    \end{enumerate}
    Die bezüglich der euklidischen Längenmessung kürzesten Verbindungskurven zwischen Punkten der Ebene 
    \(\R^2\) sind genau die (parametrisierten) Geradensegmente.
\end{karte}

\begin{karte}{\(d_{euk}\)}
    Wir können eine Längen-Metrik auf \(\R^2\) definieren. Für 
    \(p,q\in \R^2\) sei \(W_{p,q}(\R^2)\) die Menge aller stückweise differenzierbaren 
    Kurven, die \(p\) und \(q\) verbinden. Wir setzen 
    \[ d_{euk}(p,q) := \inf_{c\in W_{p,q}(\R^2)} L_{euk}(c). \]
    \((\R^2, d_{euk})\) ist ein metrischer Raum und isometrisch zu \((\R^2, d_e)\).
\end{karte}